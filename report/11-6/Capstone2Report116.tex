\documentclass{article}
\usepackage[utf8]{inputenc}
\usepackage[margin=1in]{geometry}
\usepackage{setspace,mathtools,indentfirst,hyperref}

\doublespacing{}

\begin{document}
\noindent Nelson Batista, Max Inciong, and Francesca Truncale

\noindent Senior Project II --- Fall 2017

\noindent Professor Jianting Zhang

\noindent Report for Nov. 7

This week, we began work on parallelizing the subgraph isomorphism algorithm using OpenMP. A good chunk of this time was spend familiarizing ourselves with how exactly OpenMP is used. As a start, we have parallelized the creation of the \texttt{possible\_assignments} array, which is a 2-dimensional array where \texttt{possible\_assignments[i][j]} is true if and only if a possible mapping from vertex \texttt{i} in the subgraph to vertex \texttt{j} in the search graph exists. Creating this array initially consists of setting each vertex in the search graph with degree greater than or equal to a particular subgraph vertex's degree as a possible assignment for that vertex. This is a relatively expensive process, but one which we have attempted to parallelize and are in the process of testing.

We have also reworked the entirety of the graph class with a much clearer idea of what exactly we need from such a class. The new graph class is perfectly tailored to our specific needs, and is very easy to use, since much of the common functionality has been wrapped into class methods, which in turn call upon other class methods for modification of class members when possible. This will make code much easier to write and bugs much easier to avoid, since proper testing can easily eliminate or confirm the graph class's interfaces as the source of a particular bug.

\end{document}
